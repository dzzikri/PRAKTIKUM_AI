\chapter{Methods}

\section{The data}
PLease tell where is the data come from, a little brief of company can be put here.

\section{Method 1}
Definition, steps, algoritm or equation of method 1 and how to apply into your data
\section{Method 2}
Definition, steps, algoritm or equation of method 2 and how to apply into your data


\section{Aip Suprapto Munari/1164063}
\subsection{Teori}
Tugas Harian 5 
\begin{enumerate}
\item Random Forest Dan Ilustrasi Gambarnya
\begin{itemize}
\item Pengertian Random Forest:
\par Random Forest adalah suatu algoritma yang digunakan pada klasifikasi data dalam jumlah yang besar. Klasifikasi random forest dilakukan melalui penggabungan pohon  dengan melakukan training pada sampel data yang dimiliki. Penggunaan pohon (tree) yang semakin banyak akan mempengaruhi akurasi yang akan didapatkan menjadi lebih baik. Penentuan klasifikasi dengan random forest diambil berdasarkan hasil voting dari pohon yang terbentuk. Pemenang dari pohon yang terbentuk ditentukan dengan vote terbanyak. 
\par Pembangunan pohon  pada random forest sampai dengan mencapai ukuran maksimum dari pohon data. Akan tetapi,pembangunan pohon random foresttidak dilakukan pemangkasan  yang merupakan sebuah metode untuk mengurangi kompleksitas ruang.
\item Ilustrasi Gambar Random Forest :
\par

\begin{figure}[ht]
\centering
\includegraphics[scale=0.9]{figures/AIP/asm1.PNG}
\caption{Random Forest}
\label{contoh}
\end{figure}

\par
\end{itemize}

\item Cara Membaca Dataset

Berikut adalah cara membaca dataset :
\begin{itemize}
\item Buka Anaconda Navigator lalu jalankan Syder, kemudian import libraries yang dibutuhkan.
\item Masukkan kode python untuk membaca file csv, lalu jalankan

\begin{figure}[ht]
\centering
\includegraphics[scale=0.5]{figures/AIP/y1.PNG}
\caption{(b)}
\label{contoh}
\end{figure}
\par (c) Maka pada window console akan menampilkan pesan berikut :
\begin{figure}[ht]
\centering
\includegraphics[scale=0.9]{figures/AIP/y2.PNG}
\caption{(c)}
\label{contoh}
\end{figure}
\par (d) Dari explorer dapat terlihat dataset yang terimport.
\begin{figure}[ht]
\centering
\includegraphics[scale=0.6]{figures/AIP/y3.PNG}
\caption{(d)}
\label{contoh}
\end{figure}
\par (e) Lalu klik dataset cell, maka akan muncul seperti berikut :
\begin{figure}[ht]
\centering
\includegraphics[scale=0.9]{figures/AIP/y4.PNG}
\caption{(e)}
\label{contoh}
\end{figure}
\par (f) Seperti yang terlihat pada gambar tersebut dataset ini memiliki Kolom Country, Age, dan Salary sebagai independent variable-nya dan kolom Purchased sebagai dependent variable-nya.
\par (g) Selanjutnya buat 2 matrix of features yang berisi values dari independent variable dan dependent variable.
\par (h) Lalu tuliskan perintah berikut :
\begin{figure}[ht]
\centering
\includegraphics[scale=0.9]{figures/AIP/y5.PNG}
\caption{(h)}
\label{contoh}
\end{figure}
\par (i) Perintah yang telah dibuat di atas akan membuat sebuah global environment baru dan muncul dataset.
\par (j) Klik dataset tersebut maka muncul tabel berisi dataset.

\end{itemize}

\item Cross Validation
\begin{itemize}
\item Pengertian Cross Validation : 
\par Cross Validation adalah sebuah teknik validasi model yang digunakan untuk menilai bagaimana hasil analisis statistik akan digeneralisasi ke kumpulan data independen. Cross validation digunakan dengan tujuan prediksi, dan bila kita ingin memperkirakan seberapa akurat model model prediksi yang dilakukan dalam sebuah praktek. Tujuan dari cross validation yaitu untuk mendefinisikan dataset guna menguju dalam fase pelatihan untuk membatasi masalah seperti overfitting dan underfitting serta mendapatkan wawasan tentang bagaimana model akan digeneralisasikan ke set data independen.

\par
\end{itemize}
\item Penjelasan / Maksud Dari Score pada :
\begin{itemize}
\item Random forest ( 44\% )
\par Maksud arti score 44\%  pada random forest adalah hasil dari akurasi. Yang menggunakan 5 buah atribut yaitu dari 5 baris pertama dari set pelatihan yang akan memprediksi spesies 10, 28, 156, 10 dan 43.
\par

\item Decision Tree ( 27\% )
\par Maksud arti score 27\% pada decission tree adalah presentasi hasil dari perhitungan dataset. Dari set tentang burung pipit. Confusion matrix memberi tau hal-hal yang diharapkan, artinya, butrung-burung yang terlihat mirip saling bingung satu sama lain. 
\par

\item SVM ( 29\% )
\par Maksud arti score 29\% dari SVM adalah hasil pendekatan jaringan saraf. Di sini, akurasinya adalah 27\%, yang kurang dari akurasi 44\% sebelumnya. Oleh karena itu, dessicion tree menjadi  lebih buruk. Jika kita menggunakan Support Vector Machine (SVM), yang merupakan neural pendekatan jaringan, outputnya 29\%. Jadi 29\% pada SVM merupakan hasil otputannya.
\par
\par Hasil tersebut didapat dari hasil valdasi silang untuk memastikan bahwa membagi training test dengan cara yang berbeda. Sehingga didapat outputnya 44\% untuk hutan acak, 27\% untuk pohon keputusan, dan 29\% untuk SVM.
\par
\end{itemize}

\par
\item Confusion Matrix Dan Ilustrasinya
\begin{itemize}
\item Cara Membaca Confusion Matrix :
\par Perhitungan confusion matrix adalah sebagai berikut, akan saya beri contoh sederhana yaitu pengambilan keputusan untuk mendapatkan bantuan beasiswa. Saya menggunakan dua atribut, yaitu rekening listrik dan gaji. Yang pertama kita lakukan yaitu mencari 4 nilai yaitu a,b,c, dan d:
\par a= 4
\par b= 1
\par c= 1
\par d= 2
\par Kemudian kita dapat mencari nilai Recall, Precision, accuracy dan Error Rate
\par Recall =2/(1+2) = 0,66
\par Precision = 2/(1+2) = 0,66
\par Accuracy =(4+2)/(4+1+1+2) = 0,75
\par Error Rate =(1+1)/(4+1+1+4) = 0,2
\par Ilustrasi Confusion Matrix :
\par
\begin{figure}[ht]
\centering
\includegraphics[scale=1]{figures/AIP/asm3.PNG}
\caption{Confussion Matrik}
\label{contoh}
\end{figure}
\end{itemize}

\par
\par
\item Voting Random Forest Dan Ilustrasi Gambarnya.
\par
\begin{itemize}
\item Pengertian Voting pada Random Forest	:
\par Metode ensemble dapat mencapai akurasi tinggi dengan membangun beberapa pengklasifikasi dan menjalankan
masing-masing secara mandiri. Ketika classifier membuat sebuah keputusan, kamu dapat memanfaatkan yang terbaik
keputusan umum dan rata-rata. Jika kita menggunakan metode yang paling umum, itu disebut voting.
\item Ilustrasi Gambar Voting Random Forest :
\begin{figure}[ht]
\centering
\includegraphics[scale=1]{figures/AIP/asm2.PNG}
\caption{Voting Random forest}
\label{contoh}
\end{figure}
\end{itemize}
\end{enumerate}

\subsection{Praktek}
\begin{enumerate}
\item Aplikasi Sederhana Menggunakan Pandas
	\begin{figure}[ht]
	\centering
	\includegraphics[scale=0.5]{figures/AIP/ai1.PNG}
	\caption{Aplikasi Pandas}
	\label{contoh}
	\end{figure}
	\par Penjelasan kodingan :
		\begin{enumerate}
		\item Memanggil library.
		\item Membaca dari file pandas.
		\item Menampilkan hasil
		\end{enumerate}
	\par Sehingga menghasilkan :
	\begin{figure}[ht]
	\centering
	\includegraphics[scale=0.5]{figures/AIP/ai2.PNG}
	\caption{Hasil Pandas}
	\label{contoh}
	\end{figure}
\item Aplikasi Sederhana Menggunakan Numpy
	\begin{figure}[ht]
	\centering
	\includegraphics[scale=0.5]{figures/AIP/ai3.PNG}
	\caption{Aplikasi Numpy}
	\label{contoh}
	\end{figure}
	\par Penjelasan kodingan :
		\begin{enumerate}
		\item Memanggil library
		\item Membuat variable dengan value linspace
		\item Menampilkan hasil value
		\end{enumerate}
	\par Sehingga menghasilkan :
	\begin{figure}[ht]
	\centering
	\includegraphics[scale=0.5]{figures/AIP/ai4.PNG}
	\caption{Hasil Numpy}
	\label{contoh}
	\end{figure}
\item Aplikasi Sederhana Menggunakan Matplotlib
	\begin{figure}[ht]
	\centering
	\includegraphics[scale=0.5]{figures/AIP/ai5.PNG}
	\caption{Aplikasi Matplotlib}
	\label{contoh}
	\end{figure}
	\par Penjelasan kodingan :
		\begin{enumerate}
		\item Memanggil library
		\item Membuat variable yang berisi bahasa pemrograman
		\item Membuat variable dengan value
		\item Membuat variable untuk menentukan garis 
		\item Membuat garis koordinat
		\item Menampilkan hasil
		\end{enumerate}
	\par Sehingga menghasilkan :
	\begin{figure}[ht]
	\centering
	\includegraphics[scale=0.5]{figures/AIP/ai6.PNG}
	\caption{Hasil Matplotlib}
	\label{contoh}
	\end{figure}
\item Program Klasifikasi Random Fores
	\begin{itemize}
		\item Yang pertama dataset akan dibaca.
			\begin{figure}[ht]
			\centering
			\includegraphics[scale=0.5]{figures/AIP/ai7.PNG}
			\caption{Membaca Data File}
			\label{contoh}
			\end{figure}
		 \item Selanjutnya sebagian data awal akan dilihat dengan menggunkan listing.
		 	\begin{figure}[ht]
			\centering
			\includegraphics[scale=0.5]{figures/AIP/ai8.PNG}
			\caption{Melihat Data Sebagian}
			\label{contoh}
			\end{figure}
		\item Selanjutnya jumlah data dilihat dengan menggunakan listing.
			\begin{figure}[ht]
			\centering
			\includegraphics[scale=0.5]{figures/AIP/ai9.PNG}
			\caption{Melihat Jumlah Data}
			\label{contoh}
			\end{figure}
		\item Lalu atribut diubah menjadi kolom dengan menggunakan perintah pivot.
			\begin{figure}[ht]
			\centering
			\includegraphics[scale=0.5]{figures/AIP/ai10.PNG}
			\caption{Mengubah menjadi kolom}
			\label{contoh}
			\end{figure}
		\item Selanjutnya atribut yang telah diubah, sebagian data awalnya akan dilihat dengan menggunkan listing kembali.
			\begin{figure}[ht]
			\centering
			\includegraphics[scale=0.5]{figures/AIP/ai11.PNG}
			\caption{Lihat sebagian data awal}
			\label{contoh}
			\end{figure}
		\item Selanjutnya atribut yang telah diubah, jumlah data dilihat dengan menggunakan listing kembali.
			\begin{figure}[ht]
			\centering
			\includegraphics[scale=0.5]{figures/AIP/ai12.PNG}
			\caption{Melihat jumlah data}
			\label{contoh}
			\end{figure}
		\item Lalu mengelompokkan burung kedalam spesies yang sama dengan dua kolom imgid dan label.
			\begin{figure}[ht]
			\centering
			\includegraphics[scale=0.5]{figures/AIP/ai13.PNG}
			\caption{Mengelompokkan burung}
			\label{contoh}
			\end{figure}
		\item Lalu melakukan pivot dimana imgid menjadi index.
			\begin{figure}[ht]
			\centering
			\includegraphics[scale=0.5]{figures/AIP/ai14.PNG}
			\caption{Melalukan pivot}
			\label{contoh}
			\end{figure}
		\item Selanjutnya imgid, sebagian data awalnya akan dilihat dengan menggunkan listing untuk mengecek data.
			\begin{figure}[ht]
			\centering
			\includegraphics[scale=0.5]{figures/AIP/ai15.PNG}
			\caption{Melihat data awal imgid}
			\label{contoh}
			\end{figure}
		\item Selanjutnya imgid, jumlah data dilihat dengan menggunakan listing untuk mengecek data.
			\begin{figure}[ht]
			\centering
			\includegraphics[scale=0.5]{figures/AIP/ai16.PNG}
			\caption{Melihat jumlah data imgid}
			\label{contoh}
			\end{figure}
		\item Lalu melakukan join karena isi datanya adalah sama di antara dua data. Sehingga mendapatkan data ciri labelnya sehingga bisa dikategorikan.
			\begin{figure}[ht]
			\centering
			\includegraphics[scale=0.5]{figures/AIP/ai17.PNG}
			\caption{Data ciri label dari join}
			\label{contoh}
			\end{figure}
		\item Kemudian label yang didepan di drop dan berikan label pada data yang telah dilakukan join dengan perintah listing.
			\begin{figure}[ht]
			\centering
			\includegraphics[scale=0.5]{figures/AIP/ai18.PNG}
			\caption{Mengubah menjadi kolom}
			\label{contoh}
			\end{figure}
		\item Lalu cek kembali isinya dengan perintah listing. 
			\begin{figure}[ht]
			\centering
			\includegraphics[scale=0.5]{figures/AIP/ai19.PNG}
			\caption{Melihat isi data frame}
			\label{contoh}
			\end{figure}
		\item Kemudian data dibagi menjadi dua bagian, dimana 8000 row pertama merupakan data training dan sisanya adalah data testing.
			\begin{figure}[ht]
			\centering
			\includegraphics[scale=0.5]{figures/AIP/ai20.PNG}
			\caption{Membagi data}
			\label{contoh}
			\end{figure}
		\item Kelas random forest selanjuknya dipanggil dengan RandomForestClassifier, dengan banyak kolom yang telah ditentukan oleh max feature.
			\begin{figure}[ht]
			\centering
			\includegraphics[scale=0.5]{figures/AIP/ai21.PNG}
			\caption{Kelas Random Forest}
			\label{contoh}
			\end{figure}
		\item Kemudian untuk membangun random forest dilakukan perintah fitting dengan maksimum fitur sebanyak 50.
			\begin{figure}[ht]
			\centering
			\includegraphics[scale=0.5]{figures/AIP/ai22.PNG}
			\caption{Membangun Random forest}
			\label{contoh}
			\end{figure}
		\item Kemudian lihat hasilnya dengan perintah predict.
			\begin{figure}[ht]
			\centering
			\includegraphics[scale=0.5]{figures/AIP/ai23.PNG}
			\caption{Melihat hasil}
			\label{contoh}
			\end{figure}
		\item Lalu akan terlihat hasil score dari klasifikasi.
			\begin{figure}[ht]
			\centering
			\includegraphics[scale=0.5]{figures/AIP/ai24.PNG}
			\caption{Lihat hasil score}
			\label{contoh}
			\end{figure}
	\end{itemize}
\item Program Klasifikasi Confusion Matrix
	\begin{itemize}
		\item Setelah melakukan random forest kemudian dipetakan ke dalam confusion matrix.
			\begin{figure}[ht]
			\centering
			\includegraphics[scale=0.5]{figures/AIP/ai25.PNG}
			\caption{Memetakan ke confusion matrix}
			\label{contoh}
			\end{figure}
		\item Lalu melihat hasilnya.
			\begin{figure}[ht]
			\centering
			\includegraphics[scale=0.5]{figures/AIP/ai26.PNG}
			\caption{Melihat hasil}
			\label{contoh}
			\end{figure}
		\item Kemudian dilakukan perintah plot.
			\begin{figure}[ht]
			\centering
			\includegraphics[scale=0.5]{figures/AIP/ai27.PNG}
			\caption{Melakukan Plot}
			\label{contoh}
			\end{figure}
		\item Selanjutnya nama data akan di set agar plot sumbunya sesuai.
			\begin{figure}[ht]
			\centering
			\includegraphics[scale=0.5]{figures/AIP/ai28.PNG}
			\caption{Plotting nama data}
			\label{contoh}
			\end{figure}
		\item Setelah label berubah, maka dilakukan perintah plot.
		\begin{figure}[ht]
			\centering
			\includegraphics[scale=0.5]{figures/AIP/ai29.PNG}
			\caption{Melakukan perintah plot}
			\label{contoh}
			\end{figure}
	\end{itemize}
\item Program Klasifikasi SVM dan Decision Tree
	\begin{enumerate}
		\item Program Decision Tree
			\par Mengklasifikasikan dataset yang sama menggunakan decision tree.
				\begin{figure}[ht]
				\centering
				\includegraphics[scale=0.5]{figures/AIP/ai30.PNG}
				\caption{Klasifkasi menggunakan decision tree}
				\label{contoh}
				\end{figure}
		\item Program Klasifikasi SVM
			\par Mengklasifikasikan dataset yang sama menggunakan SVM.
				\begin{figure}[ht]
				\centering
				\includegraphics[scale=0.5]{figures/AIP/ai31.PNG}
				\caption{Klasifikasi menggunakan SVM}
				\label{contoh}
				\end{figure}
	\end{enumerate}
\item Program Cross Validation
	\begin{itemize}
		\item Melakukan pengecekan cross validation untuk random forest.
			\begin{figure}[ht]
			\centering
			\includegraphics[scale=0.5]{figures/AIP/ai32.PNG}
			\caption{Pengecekan cross validation random forest}
			\label{contoh}
			\end{figure}
		\item Melakukan pengecekan cross validation untuk decission tree.
			\begin{figure}[ht]
			\centering
			\includegraphics[scale=0.5]{figures/AIP/ai33.PNG}
			\caption{Pengecekan cross validation decision tree}
			\label{contoh}
			\end{figure}
		\item Melakukan pengecekan cross validation untuk SVM.
			\begin{figure}[ht]
			\centering
			\includegraphics[scale=0.5]{figures/AIP/ai34.PNG}
			\caption{Pengecekan cross validation SVM}
			\label{contoh}
			\end{figure}
	\end{itemize}
\item Program Pengamatan Komponen Informasi
	\begin{itemize}
		\item Melakukan pengamatan komponen informasi untuk menetahui berapa banyak tree yang dibuat, atribut yang dipakai, dan informasi lainnya.
			\begin{figure}[ht]
			\centering
			\includegraphics[scale=0.5]{figures/AIP/ai35.PNG}
			\caption{Pengamatan Komponen}
			\label{contoh}
			\end{figure}
		\item Melakukan plot informasi agar bisa dibaca.
			\begin{figure}[ht]
			\centering
			\includegraphics[scale=0.5]{figures/AIP/ai36.PNG}
			\caption{Plot informasi}
			\label{contoh}
			\end{figure}
	\end{itemize}
\end{enumerate}

\subsection{Penanganan Error}
\begin{enumerate}
\item Skrinsut Error
	\begin{figure}[ht]
	\centering
	\includegraphics[scale=0.5]{figures/AIP/ai37.PNG}
	\caption{Skrinsut Error}
	\label{contoh}
	\end{figure}
\item Tuliskan kode eror dan jenis errornya
	\par Kode Error = FileNotFoundError: File b'data/CUB 200 2011/attributes/image attribute labels . txt' does not exist
	\par Jenis Error = File not found
\item Solusi Pemecahan Masalah Error
\par Solusi dari error yang terjadi pada nomor 1 adalah perbaiki alamat direktorinya sebagai berikut :
	\begin{figure}[ht]
	\centering
	\includegraphics[scale=0.5]{figures/AIP/ai38.PNG}
	\caption{Penyelesaian}
	\label{contoh}
	\end{figure}
\par Sehingga didapat hasil seperti berikut :
	\begin{figure}[ht]
	\centering
	\includegraphics[scale=0.5]{figures/AIP/ai39.PNG}
	\caption{Hasil}
	\label{contoh}
	\end{figure}
\end{enumerate}

section{Andi Muhammad Aslam/1164064}
\begin{enumerate}

\item Random Forest merupakan algoritma yang digunakan terhadapap klasifikasi data dalam jumlah yang besar. Klasifikasi pada random forest dilakukan dengan penggabungan dicision tree dengan melakukan training terhadap sempel data yang dimiliki. Pembentukan decision tree menggunakan sample data berupa variable secara acak lalu menjalankan klasifikasi pada semua tree yang terbentuk. Random forest berupa Decision Tree agar dapat melakukan proses seleksi. Decision tree yang di buat dibagi secara strategis dari data pada kelas yang sama. Pemecahan digunakan untuk membagi data berdasarkan jenis atribut yang digunakan..  \ref{Andi}

\begin{figure}[ht]
	\centerline{\includegraphics[width=1\textwidth]{figures/andi/decision tree.jpg}}
	\caption{Random Forest.}
	\label{contoh}
	\end{figure}

\item Download dataset terdahulu kemudian buka software spyder untuk melihat isi dataset. Data yang di download berupa extensi file bernama .txt yang terdapat class dari field. Contohnya pada data jenis burung memiliki file index dan angka, dimana index berisi angka yang memiliki makna berupa jenis burung atau bahkan nama burung sedangkan field memiliki isi nilai berupa 0 dan 1 yang dimana sifatnya boolean, Ya dan Tidak. Hal ini dikarenakan komputer hanya dapat membaca bilangan biner maka dari itu field yang di isikan berupa angka. Artinya angka 0 berarti tidak dan angka 1 berarti Ya.

\item Cross validation adalah metode statistik yang digunakan untuk memperkirakan keterampilan model pembelajaran mesin. Ini biasanya digunakan dalam pembelajaran mesin yang diterapkan untuk membandingkan dan memilih model untuk masalah pemodelan prediktif yang diberikan karena mudah dipahami, mudah diimplementasikan, dan menghasilkan estimasi keterampilan yang umumnya memiliki bias lebih rendah daripada metode lainnya.

\item Penjelasan Score
	\begin{itemize}
		\item Pada score 44\% pada random forest berupa hasil akurasi.
		\item Pada score 27\% pada decision tree adalah presentasi hasil dari perhitungan dataset.
		\item Pada  score 29\% dari SVM adalah hasil pendekatan neural network.
		\item Hasil tersebut didapat dari hasil valdasi silang untuk memastikan bahwa membagi  training test dengan cara yang berbeda. Sehingga dapat diketahui hasi output yaitu 44\% untuk hutan, 27\% untuk pohon keputusan, dan 29\% untuk SVM.
	\end{itemize}

\item Untuk membaca confusion matriks dapat menggunakan source code berikut :
	\begin{verbatim}
		import numpy as np
		np.set_printoptions(precision=2)
		plt.figure(figsize=(60,60), dpi=300)
		plot_confusion_matrix(cm, classes=birds, normalize=True)
		plt.show()
	\end{verbatim}

Dimana numpy dapat mengelola data yang berhubungan pada matrix. Pada perintah code tersebut digunakan dalam melakukan read pada dataset burung dengan menggunakan metode confusion matrix. Dalam confusion matrix memiliki 4 istilah yaitu True Positive yang merupakan data posotif yang terditeksi benar, True Negatif yang merupakan data negatif akan tetapi terditeksi benar, False Positif merupakan data negatif namun terditeksi sebagai data positif, False Negatif merupakan data posotif namun terditeksi sebagai data negatif. Adapun contoh hasil read dataset menggunakan confusion matrix dapat dilihat pada figure \ref{Andi}

\item Untuk mengetahui confusion matriks kita dapat melihat contoh klasifikasi dari biner berikut ini :
	\begin{figure}[ht]
	\centerline{\includegraphics[width=1\textwidth]{figures/andi/CM.PNG}}
	\caption{Tabel Confusion Matriks}
	\label{contoh}
	\end{figure}
\item Voting merupakan proses pemilihan dari tree yang dimana akan dimunculkan hasilnya dan disimpulkan menjadi informasi yang pasti.
	\begin{figure}[ht]
	\centerline{\includegraphics[width=1\textwidth]{figures/andi/Voting.PNG}}
	\caption{Voting}
	\label{Contoh Voting}
	\end{figure}

Pada figure Voting terdapat Decision Tree yang terbagi menjadi 3 Branch yaitu tree 1, tree 2, dan tree 3. Pada tree tersebut akan dilakukan proses voting. Pada  masing-masing tree tersebut memiliki data-data yang berbeda, yang di mana data tersebut akan di pilih dengan cara voting. Hasli voting dari setiap tree tersebut menunjukkan data pada setiap tree, Di sini kita dapat menghitung akurasi dengan menambahkan angka secara diagonal, sehingga ini semua adalah contoh yang diklasifikasikan dengan benar, dan membagi jumlah tersebut dengan jumlah semua angka dalam matriks.

\end{enumerate}

\subsection{Praktek}
\begin{enumerate}
\item Aplikasi Sederhana Pandas dan Penjelasan Code
\begin{itemize}
\item Pandas:
\par 
\par
\begin{itemize}
\item Penjelasan Baris  1 : import pandas as Meong merupakan import dari library pandas dari python menggunakan inisial Meong.
\par
\item Penjelasan Baris 2 : variable a sebagai pendefinisian data yang dibuat pada baris pertama yaitu Meong, sedangkan Series merupakan sebuah array satu dimensi yang memiliki label dan digunakan untuk meyimpan data yang beragam seperti integer,string,float, dan lain sebagainya. angka pada Series yang saya gunakan yaitu ([10,20,30,40,np.nan,50]) untuk menampilkan secara default, pandas secara otomatis memberi index pada setiap baris dari series.
\par
\item Penjelasan Baris 3 : fungsi print(am) untuk mencetak atau menampilkan objek 
\par
\end{itemize}
\item Hasil:
\par
\par
\begin{figure}[ht]
\centering
\includegraphics[scale=0.8]{figures/andi/Pandas.PNG}
\caption{Pandas}
\label{contoh}
\end{figure}
\par

\par
\par
\item Aplikasi Sederhana Numpy dan Penjelasan Code 
\begin{itemize}
\item Code Numpy:
\par 
\par
\begin{itemize}
\item Penjelasan  Baris 1 : import numpy as np. untuk import library numpy dari python menggunakan inisial Kucing.
\par
\item Penjelasan  Baris 2 : matrix one np.eye3. matrix dapat dikatakan sebagai list dua dimensi dimana suatu list berisi list lagi. sedangkan eye 3 untuk menghasilkan matriks identitas dengan dimensi yang ditetapkan.
\par
\item Penjelasan  Baris 3 : matrix one. matrix pada baris ketiga ini untuk menampilkan objek numpy.
\par
\end{itemize}
\item Hasil:
\par
\par
\begin{figure}[ht]
\centering
\includegraphics[scale=0.8]{figures/andi/Numpy.PNG}
\caption{Numpy}
\label{contoh}
\end{figure}
\par
\par

\par
\par
\item Aplikasi Sederhana Matplotlib dan Penjelasan Code 
\begin{itemize}
\item Code Matplotlib:
\par 
\par
\begin{itemize}
\item Penjelasan  Baris 1 : import matplotlib.pyplot as plt merupakan modul python untuk menggambarkan plot 2D.
\par
\item Penjelasan  Baris 2 : plt.plot([1,2,3,4]) untuk menentukan angka-angka pada gambar plot 2D tersebut
\par
\item Penjelasan  Baris 3 : untuk menampilkan matplotlib berupa gambar plot 2D
\par
\end{itemize}
\item Hasil:
\par
\par
\begin{figure}[ht]
\centering
\includegraphics[scale=0.8]{figures/andi/Matplotlib.PNG}
\caption{Matplotlib}
\label{contoh}
\end{figure}
\par
\end{itemize}

\par
\par
\item Program Klasifikasi Random Forest dan Penjelasan :
\begin{itemize}
\item Code Random Forest 1 :
\par
\begin{figure}[ht]
\centering
\includegraphics[scale=0.7]{figures/andi/RF1.PNG}
\caption{Gambar1}
\label{contoh}
\end{figure}
\par
\begin{itemize}
\item Penjelasan Codingan di atas menghasilkan variabel baru yaitu imgatt. Terdapat 3 kolom dan 3677856 baris data.
\par 
\par
\end{itemize}
\item Code Random Forest 2 :
\par
\begin{figure}[ht]
\centering
\includegraphics[scale=0.7]{figures/andi/RF2.PNG}
\caption{Gambar2}
\label{contoh}
\end{figure}
\par
\begin{itemize}
\item Penjelasan  Codingan di atas berfungsi untuk melihat sebagian data awal dari dataset. Hasilnya terdapat pada gambar di atas setelah di eksekusi.
\par
\par
\end{itemize}
\item Code Random Forest 3 :
\par
\begin{figure}[ht]
\centering
\includegraphics[scale=0.7]{figures/andi/RF3.PNG}
\caption{Gambar3}
\label{contoh}
\end{figure}
\par
\begin{itemize}
\item Penjelasan  Codingan di atas merupakan tampilan untuk menampilkan hasil dari dataset yang telah di run atau di eksekusi. Dimana pada gambar di atas 3677856 merupakan baris dan 3 adalah kolom.
\par
\par
\end{itemize}
\item Code Random Forest 4 :
\par
\begin{figure}[ht]
\centering
\includegraphics[scale=0.7]{figures/andi/RF4.PNG}
\caption{Gambar 4}
\label{contoh}
\end{figure}
\par
\begin{itemize}
\item Penjelasan  Pada gambar di atas menmapilkan hasil dari variabel imgatt2. Dimana index nya 'imgid', kolom berisi 'attid' dan values atau nilainya berisi 'present'.
\par
\par
\end{itemize}
\item Code Random Forest 5 :
\par
\begin{figure}[ht]
\centering
\includegraphics[scale=0.7]{figures/andi/RF5.PNG}
\caption{Gambar 5}
\label{contoh}
\end{figure}
\par
\begin{itemize}
\item Penjelasan  Pada gambar di atas menmapilkan hasil dari variabel imgatt2.head. Dimana dataset nya ada 5 baris dan 312 kolom.
\par
\par
\end{itemize}
\item Code Random Forest 6 :
\par
\begin{figure}[ht]
\centering
\includegraphics[scale=0.7]{figures/andi/RF6.PNG}
\caption{Gambar 6}
\label{contoh}
\end{figure}
\par
\begin{itemize}
\item Penjelasan  Pada gambar di atas menampilkan jumlah dari baris dan kolom dari variabel imgatt2. Dimana 11788 adalah baris dan 312 adalah kolom.
\par
\par
\end{itemize}
\item Code Random Forest 7 :
\par
\begin{figure}[ht]
\centering
\includegraphics[scale=0.7]{figures/andi/RF7.PNG}
\caption{Gambar 7}
\label{contoh}
\end{figure}
\par
\begin{itemize}
\item Penjelasan  Pada gambar di atas menunjukkan load dari  jawabannya yang berisi " apakah burung tersebut ( subjek pada dataset ) termasuk dalam spesies yang mana ?. Kolom yang digunakan adalah imgid dan label, kemudian melakukan pivot yang mana imgid menjadi index yang artinya unik sehubungan dengan dataset yang telah dieksekusi.
\par
\par
\end{itemize}
\item Code Random Forest 8 :
\par
\begin{figure}[ht]
\centering
\includegraphics[scale=0.2]{figures/andi/RF8.PNG}
\caption{Gambar 8}
\label{contoh}
\end{figure}
\par
\begin{itemize}
\item Penjelasan  Pada gambar di atas menunjukkan hasil dari variabel imglabels. Dimana menampilkan dataset dari imgid dan label. Dan dapat dilihat hasilnya dari gambar di atas.
\par
\par
\end{itemize}
\item Code Random Forest 9 :
\par
\begin{figure}[ht]
\centering
\includegraphics[scale=0.7]{figures/andi/RF9.PNG}
\caption{Gambar 9}
\label{contoh}
\end{figure}
\par
\begin{itemize}
\item Penjelasan  Pada gambar di atas menunjukkan jumlah baris dan kolom dari variabel imglabels. Dimana hasil dari kodingan tersebut dapat dilihat setelah di run. 
\par
\par
\end{itemize}
\item Code Random Forest 10 :
\par
\begin{figure}[ht]
\centering
\includegraphics[scale=0.7]{figures/andi/RF10.PNG}
\caption{Gambar 10}
\label{contoh}
\end{figure}
\par
\begin{itemize}
\item Penjelasan  Pada gambar diatas dikarenakan isinya sama, maka bisa melakukan join antara dua data yang diesekusi ( yaitu ada imgatt2 dan imglabels ), sehingga pada hasilnya akan didapatkan data ciri dan data jawaban atau labelnya sehingga bisa dikategorikan/dikelompokkan sebagai supervised learning. Jadi perintah untuk menggabungkan kedua data, kemudian dilakukan pemisahan antara data set untuk training dan test pada dataset yang dieksekusi.
\par
\par
\end{itemize}
\item Code Random Forest 11 :
\par
\begin{figure}[ht]
\centering
\includegraphics[scale=0.7]{figures/andi/RF11.PNG}
\caption{Gambar 11}
\label{contoh}
\end{figure}
\par
\begin{itemize}
\item Penjelasan Pada gambar di atas menghasilkan pemisahan dan pemilihan tabel ( memisahkan dan memilih tabel ). 
\par
\par
\end{itemize}
\item Code Random Forest 12 :
\par
\begin{figure}[ht]
\centering
\includegraphics[scale=0.7]{figures/andi/RF12.PNG}
\caption{Gambar 12}
\label{contoh}
\end{figure}
\par
\begin{itemize}
\item Penjelasan Pada gambar di atas menunjukkan hasil dari variabel dtatthead. Dimana data nya dapat dilihat pada gambar diatas. Dan dataset nya terdiri dari 5 baris dan 312 kolom.
\par
\par
\end{itemize}
\item Code Random Forest 13 :
\par
\begin{figure}[ht]
\centering
\includegraphics[scale=0.7]{figures/andi/RF13.PNG}
\caption{Gambar 13}
\label{contoh}
\end{figure}
\par
\begin{itemize}
\item Penjelasan Pada gambar di atas menunjukkan hasil dari variabel dflabel.head. Dimana berisikan data dari imgid dan label. Dan hasilnya dapat dilihat pada gambar di atas.
\par
\par
\end{itemize}
\item Code Random Forest 14 :
\par
\begin{figure}[ht]
\centering
\includegraphics[scale=0.7]{figures/andi/RF14.PNG}
\label{contoh}
\end{figure}
\par
\begin{itemize}
\item Penjelasan  Pada gambar di atas merupakan pembagian dari data training dan dataset
\par
\par
\end{itemize}
\item Code Random Forest 15 :
\par
\begin{figure}[ht] 
\centering
\includegraphics[scale=0.7]{figures/andi/RF15.PNG}
\caption{Gambar 15}
\label{contoh}
\end{figure}
\par
\begin{itemize} 
\item Penjelasan  Pada gambar di atas merupakan pemanggilan kelas RandomForestClassifier. max features yang diartikan berapa banyak kolom pada setiap tree.
\par
\par
\end{itemize}
\item Code Random Forest 16 :
\par
\begin{figure}[ht]
\centering
\includegraphics[scale=0.7]{figures/andi/RF17.PNG}
\caption{Gambar 16}
\label{contoh}
\end{figure}
\par
\begin{itemize}
\item Penjelasan  Pada gambar di atas merupaka perintah untuk melakukan fit untuk membangun random forest yang sudah ditentukan dengan maksimum fitur sebanyak 50.
\par
\par
\end{itemize}
\item Code Random Forest 17 :
\par
\begin{figure}[ht]
\centering
\includegraphics[scale=0.7]{figures/andi/RF18.PNG}
\caption{Gambar 17}
\label{contoh}
\end{figure}
\par
\begin{itemize}
\item Penjelasan  Pada gambar di atas menunjukkan hasil dari cetakan variabel dftrainatt.head.
\par
\par
\end{itemize}
\item Code Random Forest 18 :
\par
\begin{figure}[ht]
\centering
\includegraphics[scale=0.7]{figures/andi/RF19.PNG}
\caption{Gambar 18}
\label{contoh}
\end{figure}
\par
\begin{itemize}
\item Penjelasan Pada gambar di atas merupakan hasil dari variabel dftestatt da dftsetlabel. Dimana hasilnya dapat dilihat dari pada gambar di atas
\par
\par
\end{itemize}

\end{itemize}


\par
\par
\item Program Aplikasi Confusion Matrix dan Penjelasan Keluarannya :
\begin{itemize}
\item Code Confusion Matrix 1 :
\par
\begin{figure}[ht]
\centering
\includegraphics[scale=0.7]{figures/andi/RF20.PNG}
\label{contoh}
\end{figure}
\par
\begin{itemize}
\item Penjelasan  Pada gambar di atas merupakan kodingan untuk import confusiion matrik dari random forest. untuk hasilnya dapat dilihat dari gambar.
\par 
\par
\end{itemize}
\item Code Confusion Matrix 2 :
\par
\begin{figure}[ht]
\centering
\includegraphics[scale=0.7]{figures/andi/RF21.PNG}
\caption{Gambar 20}
\label{contoh}
\end{figure}
\par
\begin{itemize}
\item Penjelasan  Pada gambar di atas merupakan tampilan dari variabel cm.
\par
\par
\end{itemize}
\item Code Confusion Matrix 3 :
\par
\begin{figure}[ht]
\centering
\includegraphics[scale=0.7]{figures/andi/RF22.PNG}
\caption{Gambar 21}
\label{contoh}
\end{figure}
\par
\begin{itemize}
\item Penjelasan Pada gambar di atas merupakan perintah untuk plot. Dan untuk hasilnya terpadat pada gambar di atas. 
\par
\par
\end{itemize}
\item Code Confusion Matrix 5 :
\par
\begin{figure}[ht]
\centering
\includegraphics[scale=0.7]{figures/andi/RF24.PNG}
\caption{Gambar 23}
\label{contoh}
\end{figure}
\par
\begin{itemize}
\item Penjelasan  Pada gambar di atas merupakan perintah plot dari gambar sebelumnya.
\par
\par
\par
\end{itemize}

\end{itemize}

\par
\par
\item Program Klasifikasi SVM dan Decision Tree Beserta Penjelasan Keluarannya :
\begin{itemize}
\item Code SVM :
\par
\begin{figure}[ht]
\centering
\includegraphics[scale=0.7]{figures/andi/RF25.PNG}
\caption{SVM}
\label{contoh}
\end{figure}
\par
\begin{itemize}
\item Penjelasan  Pada gambar di atas cara untuk mencoba klasikasi dengan SVM dengan dataset yang sama.
\par 
\par
\end{itemize}
\item Code Decision Tree :
\par
\begin{figure}[ht]
\centering
\includegraphics[scale=0.7]{figures/andi/RF26.PNG}
\caption{Decission Tree}
\label{contoh}
\end{figure}
\par
\begin{itemize}
\item Penjelasan : Pada gambar di atas merupakan cara untuk mencoba klasikasi dengan decission tree dengan dataset yang sama.
\par
\par
\end{itemize}
\end{itemize}



\par
\par
\item Program Cross Validation dan Penjelasan Keluarannya :
\begin{itemize}
\item Code Cross Validation 1 :
\par
\begin{figure}[ht]
\centering
\includegraphics[scale=0.7]{figures/andi/RF27.PNG}
\caption{Cross Validation 1}
\label{contoh}
\end{figure}
\par
\begin{itemize}
\item Penjelasan : Pada gambar di atas merupakan Hasil dari cross validation random forest.
\par 
\par
\end{itemize}
\item Code Cross Validation 2  :
\par
\begin{figure}[ht]
\centering
\includegraphics[scale=0.7]{figures/andi/RF28.PNG}
\caption{Cross Validation 2}
\label{contoh}
\end{figure}
\par
\begin{itemize}
\item Penjelasan : Pada gambar di atas merupakan hasil dari cross validation Decission tree.
\par
\par
\end{itemize}
\item Code Cross Validation 3 :
\par
\begin{figure}[ht]
\centering
\includegraphics[scale=0.7]{figures/andi/RF29.PNG}
\caption{Cross Validation 3}
\label{contoh}
\end{figure}
\par
\begin{itemize}
\item Penjelasan : Pada gambar di atas merupakan hasil dari cross validation SVM.
\par
\par
\end{itemize}
\end{itemize}



\par
\par
\item Program Pengamatan Komponen Informasi dan Penjelasan Keluarannya :
\begin{itemize}
\item Code Pengamatan Komponen Informasi 1 :
\par
\begin{figure}[ht]
\centering
\includegraphics[scale=0.7]{figures/andi/RF30.PNG}
\caption{Program Pengamatan Komponen Informasi 1}
\label{contoh}
\end{figure}
\par
\begin{itemize}
\item Penjelasan : Pada gambar di atas menunjukkan cara untuk mengetahui berapa banyak tree yang dibuat, berapa banyak atribut yang dipakai dan informasi lainnya menggunakan kode.
\par 
\par
\end{itemize}
\item Code Pengamatan Komponen Informasi 2 :
\par
\begin{figure}[ht]
\centering
\includegraphics[scale=0.7]{figures/andi/RF31.PNG}
\caption{Program Pengamatan Komponen Informasi 2}
\label{contoh}
\end{figure}
\par
\begin{itemize}
\item Penjelasan : Pada gambar di atas merupakan cara untuk  melakukan plot informasi ini dengan kode di atas.
\par 
\par
\end{itemize}
\end{itemize}
\end{itemize}
\end{itemize}

\item Penanganan Error
\begin{itemize}
\item Skrinsut Error
\par
\begin{figure}[ht]
\centering
\includegraphics[scale=0.7]{figures/andi/elor.PNG}
\caption{Error}
\label{contoh}
\end{figure}
\par
\begin{itemize}
\item Kode Error: file b'data/CUB 200 2011/attributes/image attributes labels.txt'
\par 
\item Solusi Pemecahan Error : Hapus Direktori data pada kode pastikan satu folder.
\par 
\par
\end{itemize}
\end{itemize}

\end{enumerate}
